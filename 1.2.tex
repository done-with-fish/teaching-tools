\documentclass[12pt]{article}

\usepackage{amsmath}
\usepackage{amssymb}
\usepackage{amsthm}
\usepackage{cite}
%\usepackage{blindtext}
\usepackage{mathtools}
\usepackage{nicefrac}
\usepackage{fullpage}
\usepackage{manuscript}
\usepackage{sagetex}
\usepackage{tikz}
\usepackage{tikz-cd}
% \usepackage{kbordermatrix}

% \everymath{\mathtt{\xdef\tmp{\fam\the\fam\relax}\aftergroup\tmp}}
% \everydisplay{\mathtt{\xdef\tmp{\fam\the\fam\relax}\aftergroup\tmp}}
% \setbox0\hbox{$ $}

\theoremstyle{plain}
\newtheorem*{thm}{Thm}

\theoremstyle{remark}
\newtheorem*{recall}{Recall}

\theoremstyle{definition}
\newtheorem*{definition}{Def}
\newtheorem*{ex}{Ex}
\newtheorem*{badidea}{Bad Idea}
\newtheorem*{goodidea}{Good Idea}
\newtheorem*{question}{Q}
\newtheorem*{answer}{Answer}
\newtheorem*{idea}{Idea}
\newtheorem*{sol}{Sol}
\newtheorem*{notation}{Notation}
\newtheorem*{note}{Note}

\providecommand\given{} % so it exists
\newcommand\SetSymbol[1][]{
  \nonscript\,#1\vert \allowbreak \nonscript\,\mathopen{}}
\DeclarePairedDelimiterX\Set[1]{\lbrace}{\rbrace}%
{ \renewcommand\given{\SetSymbol[\delimsize]} #1 }

\DeclareMathOperator{\adj}{adj}
\DeclareMathOperator{\rref}{rref}
\DeclareMathOperator{\rank}{rank}
\DeclareMathOperator{\nullity}{nullity}
\DeclareMathOperator{\ent}{ent}
\DeclareMathOperator{\GL}{GL}
\DeclareMathOperator{\diag}{diag}
\DeclareMathOperator{\Sym}{Sym}
\DeclareMathOperator{\Row}{Row}
\DeclareMathOperator{\Col}{Col}

\DeclarePairedDelimiter\abs{\lvert}{\rvert}

\setcounter{secnumdepth}{0}

\pagenumbering{gobble}

% macros for math characters
\def\AA{\mathbb{A}} \def\calA{\mathcal{A}} \def\scrA{\mathscr{A}} \def\frakA{\mathfrak{A}} \def\bA{\mathbf{A}} \def\ba{\mathbf{a}} \def\sfA{\mathsf{A}} 
\def\BB{\mathbb{B}} \def\calB{\mathcal{B}} \def\scrB{\mathscr{B}} \def\frakB{\mathfrak{B}} \def\bB{\mathbf{B}} \def\bb{\mathbf{b}} \def\sfB{\mathsf{B}}
\def\CC{\mathbb{C}} \def\calC{\mathcal{C}} \def\scrC{\mathscr{C}} \def\frakC{\mathfrak{C}} \def\bC{\mathbf{C}} \def\bc{\mathbf{c}} \def\sfC{\mathsf{C}}
\def\DD{\mathbb{D}} \def\calD{\mathcal{D}} \def\scrD{\mathscr{D}} \def\frakD{\mathfrak{D}} \def\bD{\mathbf{D}} \def\bd{\mathbf{d}} \def\sfD{\mathsf{D}}
\def\EE{\mathbb{E}} \def\calE{\mathcal{E}} \def\scrE{\mathscr{E}} \def\frakE{\mathfrak{E}} \def\bE{\mathbf{E}} \def\be{\mathbf{e}} \def\sfE{\mathsf{E}}
\def\FF{\mathbb{F}} \def\calF{\mathcal{F}} \def\scrF{\mathscr{F}} \def\frakF{\mathfrak{F}} \def\bF{\mathbf{F}} \def\bf{\mathbf{f}} \def\sfF{\mathsf{F}}
\def\GG{\mathbb{G}} \def\calG{\mathcal{G}} \def\scrG{\mathscr{G}} \def\frakG{\mathfrak{G}} \def\bG{\mathbf{G}} \def\bg{\mathbf{g}} \def\sfG{\mathsf{G}}
\def\HH{\mathbb{H}} \def\calH{\mathcal{H}} \def\scrH{\mathscr{H}} \def\frakH{\mathfrak{H}} \def\bH{\mathbf{H}} \def\bh{\mathbf{h}} \def\sfH{\mathsf{H}}
\def\II{\mathbb{I}} \def\calI{\mathcal{I}} \def\scrI{\mathscr{I}} \def\frakI{\mathfrak{I}} \def\bI{\mathbf{I}} \def\bi{\mathbf{i}} \def\sfI{\mathsf{I}}
\def\JJ{\mathbb{J}} \def\calJ{\mathcal{J}} \def\scrJ{\mathscr{J}} \def\frakJ{\mathfrak{J}} \def\bJ{\mathbf{J}} \def\bj{\mathbf{j}} \def\sfJ{\mathsf{J}}
\def\KK{\mathbb{K}} \def\calK{\mathcal{K}} \def\scrK{\mathscr{K}} \def\frakK{\mathfrak{K}} \def\bK{\mathbf{K}} \def\bk{\mathbf{k}} \def\sfK{\mathsf{K}}
\def\LL{\mathbb{L}} \def\calL{\mathcal{L}} \def\scrL{\mathscr{L}} \def\frakL{\mathfrak{L}} \def\bL{\mathbf{L}} \def\bl{\mathbf{l}} \def\sfL{\mathsf{L}}
\def\MM{\mathbb{M}} \def\calM{\mathcal{M}} \def\scrM{\mathscr{M}} \def\frakM{\mathfrak{M}} \def\bM{\mathbf{M}} \def\bm{\mathbf{m}} \def\sfM{\mathsf{M}}
\def\NN{\mathbb{N}} \def\calN{\mathcal{N}} \def\scrN{\mathscr{N}} \def\frakN{\mathfrak{N}} \def\bN{\mathbf{N}} \def\bn{\mathbf{n}} \def\sfN{\mathsf{N}}
\def\OO{\mathbb{O}} \def\calO{\mathcal{O}} \def\scrO{\mathscr{O}} \def\frakO{\mathfrak{O}} \def\bO{\mathbf{O}} \def\bo{\mathbf{o}} \def\sfO{\mathsf{O}}
\def\PP{\mathbb{P}} \def\calP{\mathcal{P}} \def\scrP{\mathscr{P}} \def\frakP{\mathfrak{P}} \def\bP{\mathbf{P}} \def\bp{\mathbf{p}} \def\sfP{\mathsf{P}}
\def\QQ{\mathbb{Q}} \def\calQ{\mathcal{Q}} \def\scrQ{\mathscr{Q}} \def\frakQ{\mathfrak{Q}} \def\bQ{\mathbf{Q}} \def\bq{\mathbf{q}} \def\sfQ{\mathsf{Q}}
\def\RR{\mathbb{R}} \def\calR{\mathcal{R}} \def\scrR{\mathscr{R}} \def\frakR{\mathfrak{R}} \def\bR{\mathbf{R}} \def\br{\mathbf{r}} \def\sfR{\mathsf{R}}
\def\SS{\mathbb{S}} \def\calS{\mathcal{S}} \def\scrS{\mathscr{S}} \def\frakS{\mathfrak{S}} \def\bS{\mathbf{S}} \def\bs{\mathbf{s}} \def\sfS{\mathsf{S}}
\def\TT{\mathbb{T}} \def\calT{\mathcal{T}} \def\scrT{\mathscr{T}} \def\frakT{\mathfrak{T}} \def\bT{\mathbf{T}} \def\bt{\mathbf{t}} \def\sfT{\mathsf{T}}
\def\UU{\mathbb{U}} \def\calU{\mathcal{U}} \def\scrU{\mathscr{U}} \def\frakU{\mathfrak{U}} \def\bU{\mathbf{U}} \def\bu{\mathbf{u}} \def\sfU{\mathsf{U}}
\def\VV{\mathbb{V}} \def\calV{\mathcal{V}} \def\scrV{\mathscr{V}} \def\frakV{\mathfrak{V}} \def\bV{\mathbf{V}} \def\bv{\mathbf{v}} \def\sfV{\mathsf{V}}
\def\WW{\mathbb{W}} \def\calW{\mathcal{W}} \def\scrW{\mathscr{W}} \def\frakW{\mathfrak{W}} \def\bW{\mathbf{W}} \def\bw{\mathbf{w}} \def\sfW{\mathsf{W}}
\def\XX{\mathbb{X}} \def\calX{\mathcal{X}} \def\scrX{\mathscr{X}} \def\frakX{\mathfrak{X}} \def\bX{\mathbf{X}} \def\bx{\mathbf{x}} \def\sfX{\mathsf{X}}
\def\YY{\mathbb{Y}} \def\calY{\mathcal{Y}} \def\scrY{\mathscr{Y}} \def\frakY{\mathfrak{Y}} \def\bY{\mathbf{Y}} \def\by{\mathbf{y}} \def\sfY{\mathsf{Y}}
\def\ZZ{\mathbb{Z}} \def\calZ{\mathcal{Z}} \def\scrZ{\mathscr{Z}} \def\frakZ{\mathfrak{Z}} \def\bZ{\mathbf{Z}} \def\bz{\mathbf{z}} \def\sfZ{\mathsf{Z}}

\begin{document}

\title{Matrices and Matrix Operations}
\author{Brian D.\ Fitzpatrick}
\date{\cite[\S1.2]{peterson}}

\maketitle

% \tableofcontents

\begin{sagesilent}
  latex.matrix_delimiters(left='[', right=']')
\end{sagesilent}

\begin{definition}
  A \emph{matrix of size $m\times n$} is an object of the form
  \begin{align*}
    A &= 
    \begin{bmatrix}
      a_{11} & a_{12} & \dotsb & a_{1n} \\
      a_{21} & a_{22} & \dotsb & a_{2n} \\
      \vdots & \vdots & \ddots & \vdots \\
      a_{m1} & a_{m2} & \dotsb & a_{mn}
    \end{bmatrix} &
    \underbrace{m}_{\#\mathtt{rows}}\times\underbrace{n}_{\#\mathtt{columns}}
  \end{align*}
  where $a_{ij}\in\RR$. 
\end{definition}

\begin{sagesilent}
  A = matrix([[2,0],[4,1],[16,-3/4]])
  B = matrix([[3,0,8,1,2],[7,-1/2,3,2,10],[32,1,0,0,0]])
\end{sagesilent}

\begin{ex}
  \begin{align*}
    A &= \sage{A} & 3\times 2 && a_{21} &= 4 \\
    B &= \sage{B} & 3\times 5 && b_{32} &= 1 
  \end{align*}
\end{ex}

\begin{notation}
  The entries of $A=[a_{ij}]$ are often denoted by
  \[
  a_{ij}=[A]_{ij}=\ent_{ij}(A)
  \]
\end{notation}

\begin{definition}
  The collection of all $m\times n$ matrices is denoted by $M_{m\times n}(\RR)$.
\end{definition}

\begin{definition}
  Matrices in $M_{1\times n}(\RR)$ are called \emph{row vectors}. Matrices in
  $M_{m\times 1}(\RR)$ are called \emph{column vectors}.
\end{definition}

\begin{ex}
  We have the identifications
  \begin{align*}
    M_{2\times 2}(\RR) &=
    \Set*{
      \begin{bmatrix}
        a_{11} & a_{12} \\
        a_{21} & a_{22}
      \end{bmatrix}\given a_{ij}\in\RR} &
    \RR^n &= M_{n\times 1} = 
    \Set*{
      \begin{bmatrix}
        a_1\\ a_2\\ \vdots \\ a_n
      \end{bmatrix}
      \given a_i\in\RR
    }
  \end{align*}
\end{ex}

\section{Matrix Addition}

\begin{definition}
  The \emph{sum} of two $m\times n$ matrices $A$ and $B$ is the $m\times n$
  matrix $A+B$ whose entries are $[A+B]_{ij}=a_{ij}+b_{ij}$.
\end{definition}

\begin{sagesilent}
  A = matrix([[3,1,8],[2,4,-1]])
  B = matrix([[0,-3,-4],[1,2,-7]])
\end{sagesilent}

\begin{ex}
  $\sage{A}+\sage{B}=\sage{A+B}$
\end{ex}

\begin{note}
  $A+B$ is only defined when $A$ and $B$ \emph{have the same size}
\end{note}

\section{Scalar Multiplication}

\begin{definition}
  The \emph{scalar product} of $\lambda\in\RR$ and $A\in M_{m\times n}(\RR)$ is
  the $m\times n$ matrix $\lambda A$ whose entries are $[\lambda A]_{ij}=\lambda
  a_{ij}$.
\end{definition}

\begin{ex}
  $6\sage{A}=\sage{6*A}$
\end{ex}

\begin{samepage}
  \begin{thm}[Theorem 1.2 in \cite{peterson}]
    Let $A,B,C\in M_{m\times n}(\RR)$ and let $\alpha,\beta\in\RR$. Then
    \begin{enumerate}
    \item $A+B=B+A$
    \item $A+(B+C)=(A+B)+C$
    \item $\alpha(\beta A)=(\alpha\beta)A$
    \item $\alpha(A+B)=\alpha A+\alpha B$
    \item $(\alpha+\beta)A=\alpha A+\beta A$
    \end{enumerate}
  \end{thm}
\end{samepage}

\begin{definition}
  The \emph{$m\times n$ zero matrix} is the $m\times n$ matrix $0_{m\times n}$
  whose entries are $[0_{m\times n}]_{ij}=0$.
\end{definition}

\begin{ex}
  $0_{3\times 4}=\sage{zero_matrix(3,4)}$
\end{ex}

\begin{thm}
  Let $A\in M_{m\times n}(\RR)$. Then
  \begin{enumerate}
  \item $A+0_{m\times n}=A$
  \item $0\cdot A=0_{m\times n}$
  \item $A-A=0_{m\times n}$
  \end{enumerate}
\end{thm}

\section{Matrix Multiplication}

\begin{definition}
  Let $A\in M_{l\times m}(\RR)$ and $B\in M_{m\times n}(\RR)$. The
  \emph{product} of $A$ and $B$ is the $l\times n$ matrix $AB$ whose entries are
  \[
  [AB]_{ij}=\sum_{k=1}^m a_{ik}b_{kj}
  \]
\end{definition}

\begin{note}
  The product $AB$ is only defined when
  \[
  \#\texttt{ columns of } A = \#\texttt{ rows of }B
  \]
\end{note}

\begin{idea}
  The $(i,j)$ entry of $AB$ is given by the dot-product
  \begin{align*}
    [AB]_{ij}
    &= (i^{\texttt{th}}\texttt{ row of }A)\cdot(j^{\texttt{th}}\texttt{ column of }B) \\
    &= 
    \begin{bmatrix}
      \\ a_{i1} & a_{i2} & \dotsb & a_{in} \\ \\
    \end{bmatrix}
    \begin{bmatrix}
      && b_{1j} && \\
      && b_{2j} && \\
      && \vdots && \\
      && b_{mj} && 
    \end{bmatrix}
  \end{align*}
\end{idea}

\begin{sagesilent}
  A = matrix([[3,4],[1,-8]])
  B = matrix([[1,0],[2,0],[3,1]])
\end{sagesilent}

\begin{ex}
  Find $AB$ and $BA$ where
  \begin{align*}
    A &= \sage{A} & B &= \sage{B}
  \end{align*}
\end{ex}
\begin{sol}
  Note that $A$ is $2\times 2$ and $B$ is $3\times 2$. Since $2\neq 3$, $AB$ is
  not defined.

  However, $BA$ is defined and
  \[
  BA=\sage{B}\sage{A}=\sage{B*A}
  \]
\end{sol}

\begin{samepage}
  \begin{thm}[Theorem 1.3 in \cite{peterson}]
    Let $A$, $B$, and $C$ be matrices and let $\lambda\in\RR$. Then
    \begin{enumerate}
    \item $A(BC)=(AB)C$
    \item $A(B+C)=AB+AC$
    \item $(A+B)C=AC+BC$
    \item $\lambda(AB)=(\lambda A)B=A(\lambda B)$
    \end{enumerate}
  \end{thm}
\end{samepage}

\begin{definition}
  The \emph{$n\times n$ identity matrix} is the $n\times n$ matrix $I_n$ whose
  entries are
  \[
  [I_n]_{ij} =
  \delta_{ij}=
  \begin{cases}
    1 & i=j \\
    0 & i\neq j
  \end{cases}
  \]
\end{definition}

\begin{ex}
  \begin{align*}
    I_2 &= \sage{identity_matrix(2)} &
    I_3 &= \sage{identity_matrix(3)} &
    I_4 &= \sage{identity_matrix(4)} 
  \end{align*}
\end{ex}

\begin{samepage}
  \begin{thm}
    Let $A\in M_{m\times n}(\RR)$. Then
    \begin{enumerate}
    \item $0_{l\times m}A=0_{l\times n}$ and $A0_{n\times l}=0_{m\times l}$
    \item $I_mA=AI_n=A$
    \end{enumerate}
  \end{thm}
  \begin{proof}
    To prove that $I_mA=A$, note that
    \[
    [I_mA]_{ij}=\sum_{k=1}^m[I_m]_{ik}a_{kj}=\sum_{k=1}^m\delta_{ik}a_{kj}=a_{ij}
    \]
    Proving that $AI_n=A$ is similar.
  \end{proof}
\end{samepage}

The previous two theorems show that matrix arithmetic is similar to ''usual''
arithmetic. However, there are some key differences.

\begin{sagesilent}
  A = matrix([[0,1],[0,0]])
  B = matrix([[1,0],[0,0]])
\end{sagesilent}

\begin{ex}
  Find $AB$ and $BA$ where
  \begin{align*}
    A &= \sage{A} &
    B &= \sage{B}
  \end{align*}
\end{ex}
\begin{sol}
  Compute
  \begin{align*}
    AB &= \sage{A}\sage{B}=\sage{A*B} &
    BA &= \sage{B}\sage{A}=\sage{B*A}
  \end{align*}
  So $AB\neq BA$. Furthermore, $AB=0_{2\times 2}$ even though $A\neq 0_{w\times
    2}$ and $B\neq0_{2\times 2}$.
\end{sol}

\section{Motivation}

The system
\[
\begin{array}{rcrcccrcr}
  a_{11}x_1 & + & a_{12}x_2 & + & \dotsb & + & a_{1n}x_n & = & b_1 \\
  a_{21}x_1 & + & a_{22}x_2 & + & \dotsb & + & a_{2n}x_n & = & b_2 \\
  \vdots   &   & \vdots   &   & \ddots &   & \vdots   &   & \vdots \\
  a_{m1}x_1 & + & a_{m2}x_2 & + & \dotsb & + & a_{mn}x_n & = & b_m \\
\end{array}
\]
may be written as $A\vec x=\vec b$ where
\begin{align*}
  A &= 
  \begin{bmatrix}
    a_{11} & a_{12} & \dotsb & a_{1n} \\
    a_{21} & a_{22} & \dotsb & a_{2n} \\
    \vdots & \vdots & \ddots & \vdots \\
    a_{m1} & a_{m2} & \dotsb & a_{mn}
  \end{bmatrix} &
  \vec x &=
  \begin{bmatrix}
    x_1\\ x_2\\ \vdots\\ x_n
  \end{bmatrix} &
  \vec b &=     
  \begin{bmatrix}
    b_1\\ b_2\\ \vdots\\ b_m
  \end{bmatrix}
\end{align*}

\begin{sagesilent}
  A = matrix([[2,-1,4],[1,-7,1],[-1,2,1]])
  var('x y z')
  vx = matrix([[x, y, z]]).transpose()
  b = matrix([[1,3,2]]).transpose()
\end{sagesilent}

\begin{ex}
  The system
  \[
  \begin{array}{rcrcrcr}
    2\,x & - & y    & + & 4\,z & = & 1 \\
    x & - & 7\,y & + &    z & = & 3 \\
    -x & + & 2\,y & + &    z & = & 2
  \end{array}
  \]
  may be written as
  \[
  \sage{A}\sage{vx}=\sage{b}
  \]
\end{ex}

\bibliography{mybib.bib}{}
\bibliographystyle{plain}

\end{document}
