\documentclass[12pt]{article}

\usepackage{amsmath}
\usepackage{amssymb}
\usepackage{amsthm}
\usepackage{cite}
%\usepackage{blindtext}
\usepackage{mathtools}
\usepackage{nicefrac}
\usepackage{fullpage}
\usepackage{manuscript}
\usepackage{sagetex}
\usepackage{tikz}
\usepackage{tikz-cd}
% \usepackage{kbordermatrix}

% \everymath{\mathtt{\xdef\tmp{\fam\the\fam\relax}\aftergroup\tmp}}
% \everydisplay{\mathtt{\xdef\tmp{\fam\the\fam\relax}\aftergroup\tmp}}
% \setbox0\hbox{$ $}

\theoremstyle{plain}
\newtheorem*{thm}{Thm}

\theoremstyle{remark}
\newtheorem*{recall}{Recall}

\theoremstyle{definition}
\newtheorem*{definition}{Def}
\newtheorem*{ex}{Ex}
\newtheorem*{badidea}{Bad Idea}
\newtheorem*{goodidea}{Good Idea}
\newtheorem*{question}{Q}
\newtheorem*{answer}{Answer}
\newtheorem*{idea}{Idea}
\newtheorem*{sol}{Sol}
\newtheorem*{notation}{Notation}
\newtheorem*{note}{Note}

\providecommand\given{} % so it exists
\newcommand\SetSymbol[1][]{
  \nonscript\,#1\vert \allowbreak \nonscript\,\mathopen{}}
\DeclarePairedDelimiterX\Set[1]{\lbrace}{\rbrace}%
{ \renewcommand\given{\SetSymbol[\delimsize]} #1 }

\DeclareMathOperator{\adj}{adj}
\DeclareMathOperator{\rref}{rref}
\DeclareMathOperator{\rank}{rank}
\DeclareMathOperator{\nullity}{nullity}
\DeclareMathOperator{\ent}{ent}
\DeclareMathOperator{\GL}{GL}
\DeclareMathOperator{\diag}{diag}
\DeclareMathOperator{\Sym}{Sym}
\DeclareMathOperator{\Row}{Row}
\DeclareMathOperator{\Col}{Col}

\DeclarePairedDelimiter\abs{\lvert}{\rvert}

\setcounter{secnumdepth}{0}

\pagenumbering{gobble}

% macros for math characters
\def\AA{\mathbb{A}} \def\calA{\mathcal{A}} \def\scrA{\mathscr{A}} \def\frakA{\mathfrak{A}} \def\bA{\mathbf{A}} \def\ba{\mathbf{a}} \def\sfA{\mathsf{A}} 
\def\BB{\mathbb{B}} \def\calB{\mathcal{B}} \def\scrB{\mathscr{B}} \def\frakB{\mathfrak{B}} \def\bB{\mathbf{B}} \def\bb{\mathbf{b}} \def\sfB{\mathsf{B}}
\def\CC{\mathbb{C}} \def\calC{\mathcal{C}} \def\scrC{\mathscr{C}} \def\frakC{\mathfrak{C}} \def\bC{\mathbf{C}} \def\bc{\mathbf{c}} \def\sfC{\mathsf{C}}
\def\DD{\mathbb{D}} \def\calD{\mathcal{D}} \def\scrD{\mathscr{D}} \def\frakD{\mathfrak{D}} \def\bD{\mathbf{D}} \def\bd{\mathbf{d}} \def\sfD{\mathsf{D}}
\def\EE{\mathbb{E}} \def\calE{\mathcal{E}} \def\scrE{\mathscr{E}} \def\frakE{\mathfrak{E}} \def\bE{\mathbf{E}} \def\be{\mathbf{e}} \def\sfE{\mathsf{E}}
\def\FF{\mathbb{F}} \def\calF{\mathcal{F}} \def\scrF{\mathscr{F}} \def\frakF{\mathfrak{F}} \def\bF{\mathbf{F}} \def\bf{\mathbf{f}} \def\sfF{\mathsf{F}}
\def\GG{\mathbb{G}} \def\calG{\mathcal{G}} \def\scrG{\mathscr{G}} \def\frakG{\mathfrak{G}} \def\bG{\mathbf{G}} \def\bg{\mathbf{g}} \def\sfG{\mathsf{G}}
\def\HH{\mathbb{H}} \def\calH{\mathcal{H}} \def\scrH{\mathscr{H}} \def\frakH{\mathfrak{H}} \def\bH{\mathbf{H}} \def\bh{\mathbf{h}} \def\sfH{\mathsf{H}}
\def\II{\mathbb{I}} \def\calI{\mathcal{I}} \def\scrI{\mathscr{I}} \def\frakI{\mathfrak{I}} \def\bI{\mathbf{I}} \def\bi{\mathbf{i}} \def\sfI{\mathsf{I}}
\def\JJ{\mathbb{J}} \def\calJ{\mathcal{J}} \def\scrJ{\mathscr{J}} \def\frakJ{\mathfrak{J}} \def\bJ{\mathbf{J}} \def\bj{\mathbf{j}} \def\sfJ{\mathsf{J}}
\def\KK{\mathbb{K}} \def\calK{\mathcal{K}} \def\scrK{\mathscr{K}} \def\frakK{\mathfrak{K}} \def\bK{\mathbf{K}} \def\bk{\mathbf{k}} \def\sfK{\mathsf{K}}
\def\LL{\mathbb{L}} \def\calL{\mathcal{L}} \def\scrL{\mathscr{L}} \def\frakL{\mathfrak{L}} \def\bL{\mathbf{L}} \def\bl{\mathbf{l}} \def\sfL{\mathsf{L}}
\def\MM{\mathbb{M}} \def\calM{\mathcal{M}} \def\scrM{\mathscr{M}} \def\frakM{\mathfrak{M}} \def\bM{\mathbf{M}} \def\bm{\mathbf{m}} \def\sfM{\mathsf{M}}
\def\NN{\mathbb{N}} \def\calN{\mathcal{N}} \def\scrN{\mathscr{N}} \def\frakN{\mathfrak{N}} \def\bN{\mathbf{N}} \def\bn{\mathbf{n}} \def\sfN{\mathsf{N}}
\def\OO{\mathbb{O}} \def\calO{\mathcal{O}} \def\scrO{\mathscr{O}} \def\frakO{\mathfrak{O}} \def\bO{\mathbf{O}} \def\bo{\mathbf{o}} \def\sfO{\mathsf{O}}
\def\PP{\mathbb{P}} \def\calP{\mathcal{P}} \def\scrP{\mathscr{P}} \def\frakP{\mathfrak{P}} \def\bP{\mathbf{P}} \def\bp{\mathbf{p}} \def\sfP{\mathsf{P}}
\def\QQ{\mathbb{Q}} \def\calQ{\mathcal{Q}} \def\scrQ{\mathscr{Q}} \def\frakQ{\mathfrak{Q}} \def\bQ{\mathbf{Q}} \def\bq{\mathbf{q}} \def\sfQ{\mathsf{Q}}
\def\RR{\mathbb{R}} \def\calR{\mathcal{R}} \def\scrR{\mathscr{R}} \def\frakR{\mathfrak{R}} \def\bR{\mathbf{R}} \def\br{\mathbf{r}} \def\sfR{\mathsf{R}}
\def\SS{\mathbb{S}} \def\calS{\mathcal{S}} \def\scrS{\mathscr{S}} \def\frakS{\mathfrak{S}} \def\bS{\mathbf{S}} \def\bs{\mathbf{s}} \def\sfS{\mathsf{S}}
\def\TT{\mathbb{T}} \def\calT{\mathcal{T}} \def\scrT{\mathscr{T}} \def\frakT{\mathfrak{T}} \def\bT{\mathbf{T}} \def\bt{\mathbf{t}} \def\sfT{\mathsf{T}}
\def\UU{\mathbb{U}} \def\calU{\mathcal{U}} \def\scrU{\mathscr{U}} \def\frakU{\mathfrak{U}} \def\bU{\mathbf{U}} \def\bu{\mathbf{u}} \def\sfU{\mathsf{U}}
\def\VV{\mathbb{V}} \def\calV{\mathcal{V}} \def\scrV{\mathscr{V}} \def\frakV{\mathfrak{V}} \def\bV{\mathbf{V}} \def\bv{\mathbf{v}} \def\sfV{\mathsf{V}}
\def\WW{\mathbb{W}} \def\calW{\mathcal{W}} \def\scrW{\mathscr{W}} \def\frakW{\mathfrak{W}} \def\bW{\mathbf{W}} \def\bw{\mathbf{w}} \def\sfW{\mathsf{W}}
\def\XX{\mathbb{X}} \def\calX{\mathcal{X}} \def\scrX{\mathscr{X}} \def\frakX{\mathfrak{X}} \def\bX{\mathbf{X}} \def\bx{\mathbf{x}} \def\sfX{\mathsf{X}}
\def\YY{\mathbb{Y}} \def\calY{\mathcal{Y}} \def\scrY{\mathscr{Y}} \def\frakY{\mathfrak{Y}} \def\bY{\mathbf{Y}} \def\by{\mathbf{y}} \def\sfY{\mathsf{Y}}
\def\ZZ{\mathbb{Z}} \def\calZ{\mathcal{Z}} \def\scrZ{\mathscr{Z}} \def\frakZ{\mathfrak{Z}} \def\bZ{\mathbf{Z}} \def\bz{\mathbf{z}} \def\sfZ{\mathsf{Z}}

\begin{document}

\title{Determinants}
\author{Brian D.\ Fitzpatrick}
\date{\cite[\S1.5]{peterson}}

\maketitle

% \tableofcontents

\begin{sagesilent}
  latex.matrix_delimiters(left='[', right=']')
\end{sagesilent}

\begin{definition}
  The \emph{$(i,j)$-submatrix} of a $n\times n$ matrix $A$ is the
  $(n-1)\times(n-1)$ matrix obtained by deleting the $i$th row and $j$th column
  of $A$. We denote the $(i,j)$-submatrix of $A$ by $A_{ij}$.
\end{definition}

\begin{sagesilent}
  A = matrix([[1,2,3],[7,0,10],[0,-1,6]])
  A21 = matrix([[2,3],[-1,6]])
\end{sagesilent}
\begin{ex}
  $A = \sage{A}\implies A_{21} = \sage{A21}$
\end{ex}


\begin{definition}
  The \emph{determinant} of a $1\times 1$ matrix $A=[a_{11}]$ is
  $\det(A)=a_{11}$.
\end{definition}

\begin{definition}
  The \emph{determinant} of a $n\times n$ matrix $A$ is
  $\displaystyle\det(A)=\sum_{k=1}^n(-1)^{1+k}a_{1k}\det(A_{1k})$
\end{definition}

\begin{note}
  We often write $\det(A)=\abs{A}$.
\end{note}

\begin{sagesilent}
  var('a b c d')
  A = matrix([[a,b],[c,d]])
\end{sagesilent}
\begin{ex}
  Compute $\det\sage{A}$.
\end{ex}
\begin{sol}\leavevmode
  \begin{align*}
    \det(A) 
    &= \sum_{k=1}^2(-1)^{1+k}a_{1k}\det(A_{1k}) \\
    &= (-1)^{1+1}a_{11}\det(A_{11})+(-1)^{1+2}a_{12}\det(A_{12}) \\
    &= a\det([d])-b\det([c]) \\
    &= ad-bc
  \end{align*}
\end{sol}


\newpage
\begin{sagesilent}
  var('a11 a12 a13 a21 a22 a23 a31 a32 a33')
  A = matrix([[a11, a12, a13],[a21, a22, a23],[a31, a32, a33]])
\end{sagesilent}
\begin{ex}
  Compute $\det\sage{A}$.
\end{ex}
\begin{sagesilent}
  latex.matrix_delimiters(left='|', right='|')
  M = lambda i,j: A.delete_rows([i-1]).delete_columns([j-1])
\end{sagesilent}
\begin{sol}\leavevmode
  \begin{align*}
    \det(A)
    &= \sum_{k=1}^3(-1)^{1+k}a_{1k}\det(A_{1k}) \\
    &= (-1)^{1+1}a_{11}\det(A_{11})+(-1)^{1+2}a_{12}+(-1)^{1+3}a_{13}\det(A_{13}) \\
    &= a_{11}\sage{M(1,1)}-a_{12}\sage{M(1,2)}+a_{13}\sage{M(1,3)}
  \end{align*}
\end{sol}


\newpage
\begin{definition}
  The \emph{$(i,j)$-minor} of $A$ is $M_{ij}=\det(A_{ij})$. The
  \emph{$(i,j)$-cofactor} of $A$ is $C_{ij}=(-1)^{i+j}M_{ij}$.
\end{definition}

\begin{note}
  We this notation, our formula for determinants is
  \[
  \det(A)=\sum_{k=1}^n a_{1k}C_{1k}
  \]
\end{note}
\begin{sagesilent}
  latex.matrix_delimiters(left='[', right=']')
  var('a11 a12 a13 a21 a22 a23 a31 a32 a33')
  A = matrix([[a11, a12, a13],[a21, a22, a23],[a31, a32, a33]])
\end{sagesilent}
\begin{ex}
  For $A=\sage{A}$ we have%
  \begin{sagesilent}
    latex.matrix_delimiters(left='|', right='|')
    M = lambda i,j: A.delete_rows([i-1]).delete_columns([j-1])
  \end{sagesilent}%
  \begin{align*}
    C_{11} &= (-1)^{1+1}\sage{M(1,1)} =  \sage{M(1,1)} \\
    C_{12} &= (-1)^{1+2}\sage{M(1,2)} = -\sage{M(1,2)} \\
    C_{32} &= (-1)^{3+2}\sage{M(3,2)} = -\sage{M(3,2)}
  \end{align*}
\end{ex}

\newpage
\begin{note}
  The signs of the cofactors can be remembered by
  \[
  \begin{matrix}
    + & - & + & - & \cdots \\
    - & + & - & + & \cdots \\
    + & - & + & - & \cdots \\
    \vdots & \vdots & \vdots & \vdots & \ddots
  \end{matrix}
  \]
\end{note}


\begin{thm}[Laplace Cofactor Expansion Theorem, Theorem 1.16 in \cite{peterson}]
  For any $1\leq i\leq n$ we have
  \begin{align*}
    \det(A) &= \sum_{k=1}^n a_{ik}C_{ik}\tag{$i$th row expansion} \\
    \det(A) &= \sum_{k=1}^n a_{ki}C_{ki}\tag{$i$th column expansion} 
  \end{align*}
\end{thm}

\begin{idea}
  The Laplace Cofactor Expansion Theorem allows us to compute determinants by
  ``expanding'' about any row or column. This is nice because some rows or
  columns may be easier to expand about than others.
\end{idea}

\newpage

\begin{sagesilent}
  latex.matrix_delimiters(left='|', right='|')
  A = matrix([[1,2,3],[4,5,6],[7,8,9]])
  M = lambda i,j: A.delete_rows([i-1]).delete_columns([j-1])
\end{sagesilent}
\begin{ex}
  Compute $\sage{A}$.
\end{ex}
\begin{sol}
  Expansion about $\Row_2$ gives
  \[
  \sage{A}
  =(\sage{(-1)**(2+1)*A[1,0]})\sage{M(2,1)}
  +(\sage{(-1)**(2+2)*A[1,1]})\sage{M(2,2)}
  +(\sage{(-1)**(2+3)*A[1,2]})\sage{M(2,3)}
  =\sage{A.det()}
  \]
  Expansion about $\Col_3$ gives
  \[
  \sage{A}
  =(\sage{(-1)**(1+3)*A[0,2]})\sage{M(1,3)}
  +(\sage{(-1)**(2+3)*A[1,2]})\sage{M(2,3)}
  +(\sage{(-1)**(3+3)*A[2,2]})\sage{M(3,3)}
  =\sage{A.det()}
  \]
\end{sol}

\newpage
\begin{sagesilent}
  latex.matrix_delimiters(left='|', right='|')
  A = matrix([[7,-3,0,4],[0,1,0,3],[2,1,-2,-5],[0,4,0,6]])
  M = lambda X, i,j: X.delete_rows([i-1]).delete_columns([j-1])
\end{sagesilent}
\begin{ex}
  Compute $\sage{A}$.
\end{ex}
\begin{sol}
  Expansion about $\Col_3$ gives
  \[
  \sage{A}=(-2)\sage{M(A, 3, 3)}
  \]
  Then expand about $\Col_1$ to obtain
  \[
  \sage{M(A, 3, 3)} = (7)\sage{M(M(A, 3, 3), 1, 1)} = 7(6-12) = -42
  \]
  Hence
  \[
  \sage{A}=(-2)(-42)=\sage{A.det()}
  \]
\end{sol}


\newpage
\begin{thm}[Corollary 1.17 in \cite{peterson}]
  If $A$ has a row or column of zeros, then $\det(A)=0$.
\end{thm}

\begin{thm}[Corollary 1.18 in \cite{peterson}]
  The determinant of a triangular matrix is the product of its diagonal entries.
\end{thm}

\begin{thm}[Theorem 1.19 in \cite{peterson}]
  $\det(A^\top)=\det(A)$
\end{thm}

\begin{thm}[Theorem 1.20 in \cite{peterson}]\leavevmode
  \begin{enumerate}
  \item $A\xrightarrow{R_i\leftrightarrow R_j}B\implies\det(B)=-\det(A)$
  \item $A\xrightarrow{\lambda R_i\to R_i}B\implies\det(B)=\lambda\det(A)$
  \item $A\xrightarrow{R_i+\lambda R_j\to R_i}B\implies\det(B)=\det(A)$
  \end{enumerate}
\end{thm}

\begin{idea}
  Determinants play nicely with elementary row operations.
\end{idea}


\bibliography{mybib.bib}{}
\bibliographystyle{plain}

\end{document}
