\documentclass[12pt]{article}

\usepackage{amsmath}
\usepackage{amssymb}
\usepackage{amsthm}
\usepackage{cite}
%\usepackage{blindtext}
\usepackage{mathtools}
\usepackage{nicefrac}
\usepackage{fullpage}
\usepackage{manuscript}
\usepackage{sagetex}
\usepackage{tikz}
\usepackage{tikz-cd}
% \usepackage{kbordermatrix}

% \everymath{\mathtt{\xdef\tmp{\fam\the\fam\relax}\aftergroup\tmp}}
% \everydisplay{\mathtt{\xdef\tmp{\fam\the\fam\relax}\aftergroup\tmp}}
% \setbox0\hbox{$ $}

\theoremstyle{plain}
\newtheorem*{thm}{Thm}

\theoremstyle{remark}
\newtheorem*{recall}{Recall}

\theoremstyle{definition}
\newtheorem*{definition}{Def}
\newtheorem*{ex}{Ex}
\newtheorem*{badidea}{Bad Idea}
\newtheorem*{goodidea}{Good Idea}
\newtheorem*{question}{Q}
\newtheorem*{answer}{Answer}
\newtheorem*{idea}{Idea}
\newtheorem*{sol}{Sol}
\newtheorem*{notation}{Notation}
\newtheorem*{note}{Note}

\providecommand\given{} % so it exists
\newcommand\SetSymbol[1][]{
  \nonscript\,#1\vert \allowbreak \nonscript\,\mathopen{}}
\DeclarePairedDelimiterX\Set[1]{\lbrace}{\rbrace}%
{ \renewcommand\given{\SetSymbol[\delimsize]} #1 }

\DeclareMathOperator{\adj}{adj}
\DeclareMathOperator{\rref}{rref}
\DeclareMathOperator{\rank}{rank}
\DeclareMathOperator{\nullity}{nullity}
\DeclareMathOperator{\ent}{ent}
\DeclareMathOperator{\GL}{GL}
\DeclareMathOperator{\diag}{diag}
\DeclareMathOperator{\Sym}{Sym}
\DeclareMathOperator{\Row}{Row}
\DeclareMathOperator{\Col}{Col}

\DeclarePairedDelimiter\abs{\lvert}{\rvert}

\setcounter{secnumdepth}{0}

\pagenumbering{gobble}

% macros for math characters
\def\AA{\mathbb{A}} \def\calA{\mathcal{A}} \def\scrA{\mathscr{A}} \def\frakA{\mathfrak{A}} \def\bA{\mathbf{A}} \def\ba{\mathbf{a}} \def\sfA{\mathsf{A}} 
\def\BB{\mathbb{B}} \def\calB{\mathcal{B}} \def\scrB{\mathscr{B}} \def\frakB{\mathfrak{B}} \def\bB{\mathbf{B}} \def\bb{\mathbf{b}} \def\sfB{\mathsf{B}}
\def\CC{\mathbb{C}} \def\calC{\mathcal{C}} \def\scrC{\mathscr{C}} \def\frakC{\mathfrak{C}} \def\bC{\mathbf{C}} \def\bc{\mathbf{c}} \def\sfC{\mathsf{C}}
\def\DD{\mathbb{D}} \def\calD{\mathcal{D}} \def\scrD{\mathscr{D}} \def\frakD{\mathfrak{D}} \def\bD{\mathbf{D}} \def\bd{\mathbf{d}} \def\sfD{\mathsf{D}}
\def\EE{\mathbb{E}} \def\calE{\mathcal{E}} \def\scrE{\mathscr{E}} \def\frakE{\mathfrak{E}} \def\bE{\mathbf{E}} \def\be{\mathbf{e}} \def\sfE{\mathsf{E}}
\def\FF{\mathbb{F}} \def\calF{\mathcal{F}} \def\scrF{\mathscr{F}} \def\frakF{\mathfrak{F}} \def\bF{\mathbf{F}} \def\bf{\mathbf{f}} \def\sfF{\mathsf{F}}
\def\GG{\mathbb{G}} \def\calG{\mathcal{G}} \def\scrG{\mathscr{G}} \def\frakG{\mathfrak{G}} \def\bG{\mathbf{G}} \def\bg{\mathbf{g}} \def\sfG{\mathsf{G}}
\def\HH{\mathbb{H}} \def\calH{\mathcal{H}} \def\scrH{\mathscr{H}} \def\frakH{\mathfrak{H}} \def\bH{\mathbf{H}} \def\bh{\mathbf{h}} \def\sfH{\mathsf{H}}
\def\II{\mathbb{I}} \def\calI{\mathcal{I}} \def\scrI{\mathscr{I}} \def\frakI{\mathfrak{I}} \def\bI{\mathbf{I}} \def\bi{\mathbf{i}} \def\sfI{\mathsf{I}}
\def\JJ{\mathbb{J}} \def\calJ{\mathcal{J}} \def\scrJ{\mathscr{J}} \def\frakJ{\mathfrak{J}} \def\bJ{\mathbf{J}} \def\bj{\mathbf{j}} \def\sfJ{\mathsf{J}}
\def\KK{\mathbb{K}} \def\calK{\mathcal{K}} \def\scrK{\mathscr{K}} \def\frakK{\mathfrak{K}} \def\bK{\mathbf{K}} \def\bk{\mathbf{k}} \def\sfK{\mathsf{K}}
\def\LL{\mathbb{L}} \def\calL{\mathcal{L}} \def\scrL{\mathscr{L}} \def\frakL{\mathfrak{L}} \def\bL{\mathbf{L}} \def\bl{\mathbf{l}} \def\sfL{\mathsf{L}}
\def\MM{\mathbb{M}} \def\calM{\mathcal{M}} \def\scrM{\mathscr{M}} \def\frakM{\mathfrak{M}} \def\bM{\mathbf{M}} \def\bm{\mathbf{m}} \def\sfM{\mathsf{M}}
\def\NN{\mathbb{N}} \def\calN{\mathcal{N}} \def\scrN{\mathscr{N}} \def\frakN{\mathfrak{N}} \def\bN{\mathbf{N}} \def\bn{\mathbf{n}} \def\sfN{\mathsf{N}}
\def\OO{\mathbb{O}} \def\calO{\mathcal{O}} \def\scrO{\mathscr{O}} \def\frakO{\mathfrak{O}} \def\bO{\mathbf{O}} \def\bo{\mathbf{o}} \def\sfO{\mathsf{O}}
\def\PP{\mathbb{P}} \def\calP{\mathcal{P}} \def\scrP{\mathscr{P}} \def\frakP{\mathfrak{P}} \def\bP{\mathbf{P}} \def\bp{\mathbf{p}} \def\sfP{\mathsf{P}}
\def\QQ{\mathbb{Q}} \def\calQ{\mathcal{Q}} \def\scrQ{\mathscr{Q}} \def\frakQ{\mathfrak{Q}} \def\bQ{\mathbf{Q}} \def\bq{\mathbf{q}} \def\sfQ{\mathsf{Q}}
\def\RR{\mathbb{R}} \def\calR{\mathcal{R}} \def\scrR{\mathscr{R}} \def\frakR{\mathfrak{R}} \def\bR{\mathbf{R}} \def\br{\mathbf{r}} \def\sfR{\mathsf{R}}
\def\SS{\mathbb{S}} \def\calS{\mathcal{S}} \def\scrS{\mathscr{S}} \def\frakS{\mathfrak{S}} \def\bS{\mathbf{S}} \def\bs{\mathbf{s}} \def\sfS{\mathsf{S}}
\def\TT{\mathbb{T}} \def\calT{\mathcal{T}} \def\scrT{\mathscr{T}} \def\frakT{\mathfrak{T}} \def\bT{\mathbf{T}} \def\bt{\mathbf{t}} \def\sfT{\mathsf{T}}
\def\UU{\mathbb{U}} \def\calU{\mathcal{U}} \def\scrU{\mathscr{U}} \def\frakU{\mathfrak{U}} \def\bU{\mathbf{U}} \def\bu{\mathbf{u}} \def\sfU{\mathsf{U}}
\def\VV{\mathbb{V}} \def\calV{\mathcal{V}} \def\scrV{\mathscr{V}} \def\frakV{\mathfrak{V}} \def\bV{\mathbf{V}} \def\bv{\mathbf{v}} \def\sfV{\mathsf{V}}
\def\WW{\mathbb{W}} \def\calW{\mathcal{W}} \def\scrW{\mathscr{W}} \def\frakW{\mathfrak{W}} \def\bW{\mathbf{W}} \def\bw{\mathbf{w}} \def\sfW{\mathsf{W}}
\def\XX{\mathbb{X}} \def\calX{\mathcal{X}} \def\scrX{\mathscr{X}} \def\frakX{\mathfrak{X}} \def\bX{\mathbf{X}} \def\bx{\mathbf{x}} \def\sfX{\mathsf{X}}
\def\YY{\mathbb{Y}} \def\calY{\mathcal{Y}} \def\scrY{\mathscr{Y}} \def\frakY{\mathfrak{Y}} \def\bY{\mathbf{Y}} \def\by{\mathbf{y}} \def\sfY{\mathsf{Y}}
\def\ZZ{\mathbb{Z}} \def\calZ{\mathcal{Z}} \def\scrZ{\mathscr{Z}} \def\frakZ{\mathfrak{Z}} \def\bZ{\mathbf{Z}} \def\bz{\mathbf{z}} \def\sfZ{\mathsf{Z}}

\begin{document}

\title{Inverses of Matrices}
\author{Brian D.\ Fitzpatrick}
\date{\cite[\S1.3]{peterson}}

\maketitle

% \tableofcontents

\begin{sagesilent}
  latex.matrix_delimiters(left='[', right=']')
\end{sagesilent}

\begin{definition}
  The \emph{inverse} of $A\in M_{n\times n}(\RR)$ is a matrix $B\in M_{n\times
    n}(\RR)$ such that $AB=BA=I$.
\end{definition}

\begin{sagesilent}
  A = matrix([[1,2],[3,5]]) 
  B = matrix([[-5,2],[3,-1]])
\end{sagesilent}

\begin{ex}
  Let $A=\sage{A}$ and $B=\sage{B}$. Then
  \begin{align*}
    AB &= \sage{A}\sage{B} = \sage{A*B} = I \\
    BA &= \sage{B}\sage{A} = \sage{B*A} = I
  \end{align*}
  Hence $B$ is an inverse of $A$.
\end{ex}

\begin{ex}
  The matrix $0_{n\times n}$ does not have an inverse since $0_{n\times
    n}B=0_{n\times n}\neq I$ whenever $B\in M_{n\times n}(\RR)$.
\end{ex}

\begin{sagesilent}
  A = matrix([[1,1],[0,0]])
  var('b11 b12 b21 b22')
  B = matrix([[b11, b12],[b21, b22]])
\end{sagesilent}

\begin{ex}
  The matrix $A=\sage{A}$ has no inverse since
  \[
  AB = \sage{A}\sage{B} = \sage{A*B}
  \]
  whenever $B\in M_{2\times 2}(\RR)$.
\end{ex}

\begin{definition}
  A matrix is \emph{invertible} or \emph{nonsingular} if it has an inverse. A
  matrix is \emph{noninvertible} or \emph{singular} if it does not have an
  inverse.
\end{definition}

\begin{definition}
  The collection of all $n\times n$ invertible matrices is denoted by $\GL_n(\RR)$.
\end{definition}

\begin{thm}[Theorem 1.4 in \cite{peterson}]
  Inverses are unique. That is, let $A\in\GL_n(\RR)$ have inverses $B_1$ and
  $B_2$. Then $B_1=B_2$.
\end{thm}
\begin{proof}
  $B_1=B_1I=B_1(AB_2)=(B_1A)B_2=IB_2=B_2$
\end{proof}

\begin{note}
  Since inverses are unique, we denote the inverse of $A$ by $A^{-1}$.
\end{note}

\begin{question}
  How do we find $A^{-1}$?
\end{question}

\begin{answer}\leavevmode
  \begin{enumerate}
  \item Form the augmented matrix: $[A\mid I]$.
  \item Use elementary row operations to reduce $A$ into rref: $[A\mid
    I]\rightsquigarrow[\rref(A)\mid B]$.
  \item If $\rref(A)=I$, then $B=A^{-1}$. Otherwise, $A$ is not invertible.
  \end{enumerate}
\end{answer}

\begin{sagesilent}
  import gaussjordan
  A = matrix([[2,1,3],[2,1,1],[4,5,1]])
  B = identity_matrix(3)
  M = A.augment(B, subdivide=True)
\end{sagesilent}

\newpage
\begin{ex}
  Find the inverse of $A=\sage{A}$.
\end{ex}
\begin{sol}
  Row reduce
  {\allowdisplaybreaks  
    \sage{gaussjordan.latex_reduction(M)}}%
  This gives $A^{-1}=\sage{A.inverse()}$.
\end{sol}

\newpage
\begin{definition}
  A \emph{$n\times n$ elementary matrix} is a matrix obtained by performing
  exactly one elementary row operation on $I_n$.
\end{definition}


\begin{sagesilent}
  I2 = identity_matrix(2)
  E2 = elementary_matrix(2, row1=1, scale=3)
  I3 = identity_matrix(3)
  E3 = elementary_matrix(3, row1=1, row2=0, scale=-1/2)
  I4 = identity_matrix(4)
  E4 = elementary_matrix(4, row1=2, row2=3)
\end{sagesilent}

\begin{ex}
  \begin{align*}
    \sage{I2} &\xrightarrow{3\cdot R_2\to R_2}\sage{E2} \\
    \sage{I3} &\xrightarrow{R_2-\nicefrac{1}{2}\cdot R_1\to R_2}\sage{E3} \\
    \sage{I4} &\xrightarrow{R_3\leftrightarrow R_4}\sage{E4}
  \end{align*}
\end{ex}

\begin{idea}
  Multiplication by elementary matrices on the left results in performing the
  corresponding elementary row operation.
\end{idea}



\newpage
\begin{sagesilent}
  A = matrix([[1,7],[0,1/3]])
  E = elementary_matrix(2, row1=1, scale=3)
\end{sagesilent}

\begin{ex}
  Performing $3\cdot R_2\to R_2$ on $A=\sage{A}$ gives
  \[
  \sage{A}\xrightarrow{3\cdot R_2\to R_2}\sage{E*A}
  \]
  The row-operation $3\cdot R_2\to R_2$ corresponds to the elementary matrix
  $E=\sage{E}$. Computing $EA$ gives
  \[
  EA = \sage{E}\sage{A}=\sage{E*A}
  \]
\end{ex}

\begin{thm}
  Elementary matrices are invertible and their inverses are
  \begin{align*}
    [R_i\leftrightarrow R_j]^{-1} &= [R_i\leftrightarrow R_j] \\
    [\lambda\cdot R_i\to R_i]^{-1} &= [\nicefrac{1}{\lambda}\cdot R_i\to R_i] \\
    [R_i+\lambda\cdot R_j\to R_i]^{-1} &= [R_i-\lambda\cdot R_j\to R_i]
  \end{align*}
\end{thm}


\newpage
\begin{sagesilent}
  import gaussjordan
  A = matrix([[1,2],[3,5]])
  B = identity_matrix(2)
  M = A.augment(B, subdivide=True)
  E1 = elementary_matrix(2, row1=1, row2=0, scale=-3)
  E2 = elementary_matrix(2, row1=1, scale=-1)
  E3 = elementary_matrix(2, row1=0, row2=1, scale=-2)
\end{sagesilent}

\begin{ex}
  Write $A^{-1}$ as a product of elementary matrices where $A=\sage{A}$.
\end{ex}
\begin{sol}
  Row reduce
  {\allowdisplaybreaks  
    \sage{gaussjordan.latex_reduction(M)}}%
  This gives $A^{-1}=\sage{A.inverse()}$. 

  These row reductions correspond to elementary matrices
  \begin{align*}
    E_1 &= [R_2-3\cdot R_1\to R_2] = \sage{E1} \\
    E_2 &= [-R_2\to R_2] = \sage{E2} \\
    E_3 &= [R_1-2\cdot R_2\to R_1] = \sage{E3}
  \end{align*}
  This gives $\underbrace{E_3E_2E_1}_{=A^{-1}}A=\rref(A)=I$.
\end{sol}


\newpage
\begin{thm}
  Given $A, B\in\GL_n(\RR)$, let $\lambda\in\RR$ such that $\lambda\neq 0$. Then
  \begin{enumerate}
  \item $A^{-1}\in\GL_n(\RR)$ and $(A^{-1})^{-1}=A)$
  \item $\lambda\cdot A\in\GL_n(\RR)$ and $(\lambda A)^{-1}=\nicefrac{1}{\lambda}\cdot A^{-1}$
  \item $AB\in\GL_n(\RR)$ and $(AB)^{-1}=B^{-1}A^{-1}$
  \item $A^k\in\GL_n(\RR)$ and $(A^k)^{-1}=(A^{-1})^k$
  \end{enumerate}
\end{thm}

\begin{thm}[Fundamental Theorem of Invertible Matrices]
  Let $A\in M_{n\times n}(\RR)$. Then the following are equivalent.
  \begin{enumerate}
  \item $A$ is invertible
  \item $A$ is nonsingular
  \item $A\in\GL_n(\RR)$
  \item $A\vec x=\vec b$ has a unique solution for every $\vec b\in\RR^n$
  \item $A\vec x=\vec 0$ has only the trivial solution $\vec x=\vec 0$
  \item $\rref(A)=I_n$
  \item $\rank(A)=n$
  \item $A$ is a product of elementary matrices
  \end{enumerate}
\end{thm}


\newpage
\begin{thm}[Theorem 1.9 in \cite{peterson}]
  Let $A, B\in M_{n\times n}(\RR)$ such that $AB=I_n$ or $BA=I_n$. Then
  $B=A^{-1}$.
\end{thm}
\begin{proof}
  Suppose $BA=I_n$ and consider the system $A\vec x=\vec 0$. Then
  \[
  \vec 0= B\vec 0= B(A\vec x)= (BA)\vec x= I_n\vec x= \vec x
  \]
  So, the system $A\vec x=\vec 0$ only has the trivial solution. By the
  fundamental theorem, $A\in\GL_n(\RR)$ and 
  \[
  B = BI_n = B(AA^{-1}) = (BA)A^{-1} = I_nA^{-1} = A^{-1}
  \]
  The argument when $AB=I_n$ is similar.
\end{proof}


\bibliography{mybib.bib}{}
\bibliographystyle{plain}

\end{document}
