\documentclass[12pt]{article}

\usepackage{amsmath}
\usepackage{amssymb}
\usepackage{amsthm}
\usepackage{cite}
%\usepackage{blindtext}
\usepackage{mathtools}
\usepackage{nicefrac}
\usepackage{fullpage}
\usepackage{manuscript}
\usepackage{sagetex}
\usepackage{tikz}
\usepackage{tikz-cd}
% \usepackage{kbordermatrix}

% \everymath{\mathtt{\xdef\tmp{\fam\the\fam\relax}\aftergroup\tmp}}
% \everydisplay{\mathtt{\xdef\tmp{\fam\the\fam\relax}\aftergroup\tmp}}
% \setbox0\hbox{$ $}

\theoremstyle{plain}
\newtheorem*{thm}{Thm}

\theoremstyle{remark}
\newtheorem*{recall}{Recall}

\theoremstyle{definition}
\newtheorem*{definition}{Def}
\newtheorem*{ex}{Ex}
\newtheorem*{badidea}{Bad Idea}
\newtheorem*{goodidea}{Good Idea}
\newtheorem*{question}{Q}
\newtheorem*{answer}{Answer}
\newtheorem*{idea}{Idea}
\newtheorem*{sol}{Sol}
\newtheorem*{notation}{Notation}
\newtheorem*{note}{Note}

\providecommand\given{} % so it exists
\newcommand\SetSymbol[1][]{
  \nonscript\,#1\vert \allowbreak \nonscript\,\mathopen{}}
\DeclarePairedDelimiterX\Set[1]{\lbrace}{\rbrace}%
{ \renewcommand\given{\SetSymbol[\delimsize]} #1 }

\DeclareMathOperator{\adj}{adj}
\DeclareMathOperator{\rref}{rref}
\DeclareMathOperator{\rank}{rank}
\DeclareMathOperator{\nullity}{nullity}
\DeclareMathOperator{\ent}{ent}
\DeclareMathOperator{\GL}{GL}
\DeclareMathOperator{\diag}{diag}
\DeclareMathOperator{\Sym}{Sym}
\DeclareMathOperator{\Row}{Row}
\DeclareMathOperator{\Col}{Col}

\DeclarePairedDelimiter\abs{\lvert}{\rvert}

\setcounter{secnumdepth}{0}

\pagenumbering{gobble}

% macros for math characters
\def\AA{\mathbb{A}} \def\calA{\mathcal{A}} \def\scrA{\mathscr{A}} \def\frakA{\mathfrak{A}} \def\bA{\mathbf{A}} \def\ba{\mathbf{a}} \def\sfA{\mathsf{A}} 
\def\BB{\mathbb{B}} \def\calB{\mathcal{B}} \def\scrB{\mathscr{B}} \def\frakB{\mathfrak{B}} \def\bB{\mathbf{B}} \def\bb{\mathbf{b}} \def\sfB{\mathsf{B}}
\def\CC{\mathbb{C}} \def\calC{\mathcal{C}} \def\scrC{\mathscr{C}} \def\frakC{\mathfrak{C}} \def\bC{\mathbf{C}} \def\bc{\mathbf{c}} \def\sfC{\mathsf{C}}
\def\DD{\mathbb{D}} \def\calD{\mathcal{D}} \def\scrD{\mathscr{D}} \def\frakD{\mathfrak{D}} \def\bD{\mathbf{D}} \def\bd{\mathbf{d}} \def\sfD{\mathsf{D}}
\def\EE{\mathbb{E}} \def\calE{\mathcal{E}} \def\scrE{\mathscr{E}} \def\frakE{\mathfrak{E}} \def\bE{\mathbf{E}} \def\be{\mathbf{e}} \def\sfE{\mathsf{E}}
\def\FF{\mathbb{F}} \def\calF{\mathcal{F}} \def\scrF{\mathscr{F}} \def\frakF{\mathfrak{F}} \def\bF{\mathbf{F}} \def\bf{\mathbf{f}} \def\sfF{\mathsf{F}}
\def\GG{\mathbb{G}} \def\calG{\mathcal{G}} \def\scrG{\mathscr{G}} \def\frakG{\mathfrak{G}} \def\bG{\mathbf{G}} \def\bg{\mathbf{g}} \def\sfG{\mathsf{G}}
\def\HH{\mathbb{H}} \def\calH{\mathcal{H}} \def\scrH{\mathscr{H}} \def\frakH{\mathfrak{H}} \def\bH{\mathbf{H}} \def\bh{\mathbf{h}} \def\sfH{\mathsf{H}}
\def\II{\mathbb{I}} \def\calI{\mathcal{I}} \def\scrI{\mathscr{I}} \def\frakI{\mathfrak{I}} \def\bI{\mathbf{I}} \def\bi{\mathbf{i}} \def\sfI{\mathsf{I}}
\def\JJ{\mathbb{J}} \def\calJ{\mathcal{J}} \def\scrJ{\mathscr{J}} \def\frakJ{\mathfrak{J}} \def\bJ{\mathbf{J}} \def\bj{\mathbf{j}} \def\sfJ{\mathsf{J}}
\def\KK{\mathbb{K}} \def\calK{\mathcal{K}} \def\scrK{\mathscr{K}} \def\frakK{\mathfrak{K}} \def\bK{\mathbf{K}} \def\bk{\mathbf{k}} \def\sfK{\mathsf{K}}
\def\LL{\mathbb{L}} \def\calL{\mathcal{L}} \def\scrL{\mathscr{L}} \def\frakL{\mathfrak{L}} \def\bL{\mathbf{L}} \def\bl{\mathbf{l}} \def\sfL{\mathsf{L}}
\def\MM{\mathbb{M}} \def\calM{\mathcal{M}} \def\scrM{\mathscr{M}} \def\frakM{\mathfrak{M}} \def\bM{\mathbf{M}} \def\bm{\mathbf{m}} \def\sfM{\mathsf{M}}
\def\NN{\mathbb{N}} \def\calN{\mathcal{N}} \def\scrN{\mathscr{N}} \def\frakN{\mathfrak{N}} \def\bN{\mathbf{N}} \def\bn{\mathbf{n}} \def\sfN{\mathsf{N}}
\def\OO{\mathbb{O}} \def\calO{\mathcal{O}} \def\scrO{\mathscr{O}} \def\frakO{\mathfrak{O}} \def\bO{\mathbf{O}} \def\bo{\mathbf{o}} \def\sfO{\mathsf{O}}
\def\PP{\mathbb{P}} \def\calP{\mathcal{P}} \def\scrP{\mathscr{P}} \def\frakP{\mathfrak{P}} \def\bP{\mathbf{P}} \def\bp{\mathbf{p}} \def\sfP{\mathsf{P}}
\def\QQ{\mathbb{Q}} \def\calQ{\mathcal{Q}} \def\scrQ{\mathscr{Q}} \def\frakQ{\mathfrak{Q}} \def\bQ{\mathbf{Q}} \def\bq{\mathbf{q}} \def\sfQ{\mathsf{Q}}
\def\RR{\mathbb{R}} \def\calR{\mathcal{R}} \def\scrR{\mathscr{R}} \def\frakR{\mathfrak{R}} \def\bR{\mathbf{R}} \def\br{\mathbf{r}} \def\sfR{\mathsf{R}}
\def\SS{\mathbb{S}} \def\calS{\mathcal{S}} \def\scrS{\mathscr{S}} \def\frakS{\mathfrak{S}} \def\bS{\mathbf{S}} \def\bs{\mathbf{s}} \def\sfS{\mathsf{S}}
\def\TT{\mathbb{T}} \def\calT{\mathcal{T}} \def\scrT{\mathscr{T}} \def\frakT{\mathfrak{T}} \def\bT{\mathbf{T}} \def\bt{\mathbf{t}} \def\sfT{\mathsf{T}}
\def\UU{\mathbb{U}} \def\calU{\mathcal{U}} \def\scrU{\mathscr{U}} \def\frakU{\mathfrak{U}} \def\bU{\mathbf{U}} \def\bu{\mathbf{u}} \def\sfU{\mathsf{U}}
\def\VV{\mathbb{V}} \def\calV{\mathcal{V}} \def\scrV{\mathscr{V}} \def\frakV{\mathfrak{V}} \def\bV{\mathbf{V}} \def\bv{\mathbf{v}} \def\sfV{\mathsf{V}}
\def\WW{\mathbb{W}} \def\calW{\mathcal{W}} \def\scrW{\mathscr{W}} \def\frakW{\mathfrak{W}} \def\bW{\mathbf{W}} \def\bw{\mathbf{w}} \def\sfW{\mathsf{W}}
\def\XX{\mathbb{X}} \def\calX{\mathcal{X}} \def\scrX{\mathscr{X}} \def\frakX{\mathfrak{X}} \def\bX{\mathbf{X}} \def\bx{\mathbf{x}} \def\sfX{\mathsf{X}}
\def\YY{\mathbb{Y}} \def\calY{\mathcal{Y}} \def\scrY{\mathscr{Y}} \def\frakY{\mathfrak{Y}} \def\bY{\mathbf{Y}} \def\by{\mathbf{y}} \def\sfY{\mathsf{Y}}
\def\ZZ{\mathbb{Z}} \def\calZ{\mathcal{Z}} \def\scrZ{\mathscr{Z}} \def\frakZ{\mathfrak{Z}} \def\bZ{\mathbf{Z}} \def\bz{\mathbf{z}} \def\sfZ{\mathsf{Z}}

\begin{document}

\title{Systems of Linear Equations}
\author{Brian D.\ Fitzpatrick}
\date{\cite[\S1.1]{peterson}}

\maketitle

% \tableofcontents

\begin{sagesilent}
  latex.matrix_delimiters(left='[', right=']')
\end{sagesilent}

hello world!

\section{Motivation}

Consider the system of $m$ equations with $n$ variables.
\begin{equation}\label{eq:motivation}
  \begin{array}{rcrcccrcr}
    a_{11}x_1 & + & a_{12}x_2 & + & \dotsb & + & a_{1n}x_n & = & b_1    \\
    a_{21}x_1 & + & a_{22}x_2 & + & \dotsb & + & a_{2n}x_n & = & b_2    \\
    \vdots    &   & \vdots    &   & \ddots &   & \vdots    &   & \vdots \\
    a_{m1}x_1 & + & a_{m2}x_2 & + & \dotsb & + & a_{mn}x_n & = & b_m
  \end{array}\tag{$\ast$}
\end{equation}
\begin{itemize}
\item[Q1.] Does a solution exist?
\item[Q2.] How many solutions exist?
\item[Q3.] How do we find \emph{all} solutions?
\end{itemize}

\begin{ex}
  The system of $2$ equations and $3$ variables
  \[
  \begin{array}{rcrcrcr}
    x & + & y & + & z    & = & 2  \\
    x &   &   & - & 2\,z & = & 3
  \end{array}
  \]
  is solved by $(x,y,z)=(5,-4,1)$ and $(x,y,z)=(1,2,-1)$. The system has
  \emph{more than one} solution.
\end{ex}

\begin{ex}
  The system of $3$ equations and $2$ variables
  \[
  \begin{array}{rcrcrr}
    x     & + & 3\,y  & = & 2 & (1)\\
    -2\,x & + & 7\,y  & = & 8 & (2)\\
    4\,x  & - & 14\,y & = & 0 & (3)
  \end{array}
  \]
  has \emph{no solutions} because
  $
  2\cdot(2)+(3)
  =
  \begin{cases}
    0 & \text{LHS} \\
    16 & \text{RHS}
  \end{cases}
  $
\end{ex}

\begin{samepage}
  \begin{ex}
    Find all solutions to the system
    \[
    \begin{array}{rcrcrr}
      x    & + & 2\,y & = & 3 & (1) \\
      3\,x & + & 4\,y & = & 0 & (2)
    \end{array}
    \]
  \end{ex}
  \begin{badidea}
    Start deriving new equations. Equation (2) is
    \[
    y=-\frac{3}{4}\,x\tag{3}
    \]
    Plug (3) into (1) to obtain
    \[
    x+2\left(-\frac{3}{4}\,x\right)=3\Rightarrow x=-6\tag{4}
    \]
    Plug (4) into (3) to obtain
    \[
    y=\left(-\frac{3}{4}\right)(-6)=\frac{9}{2}\tag{5}
    \]
    Equations (4) and (5) then give $(x,y)=\left(-6,\nicefrac{9}{2}\right)$.
  \end{badidea}
\end{samepage}


\begin{samepage}
  \begin{question}
    Why is this a bad idea?
  \end{question}
  \begin{enumerate}
  \item Equations get messy in high dimensions.
  \item The procedure is not guaranteed to terminate.
  \end{enumerate}

  \begin{definition}
    Two systems are \emph{equivalent} if they have the same solutions.
  \end{definition}

  \begin{goodidea}
    Replace ''hard'' systems with equivalent ''easy'' systems.
  \end{goodidea}
\end{samepage}

\pagebreak
\section{Augmented Matrices and Reduced Row-Echelon Form}

\begin{definition}
  The \emph{augmented matrix} of the system \eqref{eq:motivation} is
  \[
  \left[
    \begin{array}{rrcr|r}
      a_{11} & a_{12} & \dotsb & a_{1n} & b_{1} \\
      a_{21} & a_{22} & \dotsb & a_{2n} & b_{2} \\
      \vdots & \vdots & \ddots & \vdots & \vdots \\
      a_{m1} & a_{m2} & \dotsb & a_{mn} & b_{m}
    \end{array}
  \right]
  \]
\end{definition}

\begin{sagesilent}
  A = matrix([[3,4,-8],[0,1,10],[9,7,-2]])
  b = vector([0,6,-5])
  M = A.augment(b, subdivide=True)
\end{sagesilent}

\begin{ex}
  \begin{align*}
    \begin{array}{rcrcrcr}
      3\,x & + & 4\,y & - & 8\,z  & = & 0 \\
      &   &    y & + & 10\,z & = & 6 \\
      9\,x & + & 7\,y & - & 2\,z  & = & -5
    \end{array} && \longleftrightarrow &&
    \sage{M}
  \end{align*}
\end{ex}

\begin{definition}
  A matrix is in \emph{reduced row-echelon form} (rref) if
  \begin{enumerate}
  \item any zero-rows occur at the bottom
  \item the first nonzero entry of a nonzero row is $1$ (called a \emph{pivot})
  \item every pivot occurs to the right of any previous pivots
  \item all other entries of a column containing a pivot are zero
  \end{enumerate}
\end{definition}

\begin{sagesilent}
  A1 = matrix([[1,0],[0,1]])
  b1 = vector([0,2])
  M1 = A1.augment(b1, subdivide=True)
  A2 = matrix([[1,1,0,0],[0,0,1,2],[0,0,0,0]])
  b2 = vector([3,1,0])
  M2 = A2.augment(b2, subdivide=True)
  A3 = matrix([[0,1,0,0,0],[0,0,0,1,0],[0,0,0,0,1],[0,0,0,0,0]])
  b3 = vector([0,2,0,0])
  M3 = A3.augment(b3, subdivide=True)
\end{sagesilent}

\begin{ex}
  Each of the following is in rref
  \begin{align*}
    \sage{M1} &&
    \sage{M2} &&
    \sage{M3}
  \end{align*}
\end{ex}

\begin{sagesilent}
  A1 = matrix([[1,2,0,0],[0,1,3,1],[0,0,0,1]])
  b1 = vector([3,1,2])
  M1 = A1.augment(b1, subdivide=True)
  A2 = matrix([[0,1,0,0],[1,0,1,1],[0,0,0,0]])
  b2 = vector([2,1,0])
  M2 = A2.augment(b2, subdivide=True)
  A3 = matrix([[1,0,0],[0,0,1],[0,0,1]])
  b3 = vector([3,2,0])
  M3 = A3.augment(b3, subdivide=True)
\end{sagesilent}

\begin{ex}
  Each of the following is \emph{not} in rref
  \begin{align*}
    \sage{M1} &&
    \sage{M2} &&
    \sage{M3}
  \end{align*}
\end{ex}

\begin{definition}
  A rref system is
  \begin{description}
  \item[\emph{consistent}] if the augmented column contains no pivot
  \item[\emph{inconsistent}] if the augmented column contains a pivot
  \end{description}
\end{definition}

\begin{definition}
  A column in a consistent rref matrix corresponds to a
  \begin{description}
  \item[\emph{free variable}] if it contains no pivot
  \item[\emph{dependent variable}] if it contains a pivot
  \end{description}
\end{definition}

\begin{idea}
  Solving consistent rref systems is \emph{easy}: write the dependent variables
  in terms of the free variables.
\end{idea}

\begin{sagesilent}
  A = matrix([[1,0,0,0],[0,0,1,2],[0,0,0,0]])
  b = vector([4,-1,0])
  M = A.augment(b, subdivide=True)
  var('x1 x2 x3 x4')
  u = matrix([[x1, x2, x3, x4]]).transpose()
  v = matrix([[4,x2,-1-2*x4,x4]]).transpose()
  w1 = matrix([4,0,-1,0]).transpose()
  w2 = matrix([0,1,0,0]).transpose()
  w3 = matrix([0,0,-2,1]).transpose()
\end{sagesilent}

\begin{samepage}
  \begin{ex}
    The rref system
    \[
    \sage{M}
    \]
    is consistent. The free variables are $\Set{x_2,x_4}$ and the dependent
    variables are $\Set{x_1,x_3}$. The solutions are
    \[
    \sage{u}
    =\sage{v}
    =\sage{w1}+x_2\sage{w2}+x_4\sage{w3}
    \]
    Note that there are \emph{infinitely many solutions}.
  \end{ex}
\end{samepage}

\begin{sagesilent}
  A = matrix([[1,0,0],[0,1,0],[0,0,1]])
  b = vector([3,4,8])
  M = A.augment(b, subdivide=True)
\end{sagesilent}

\begin{ex}
  The rref system
  \[
  \sage{M}
  \]
  is consistent with no free variables. The only solution is
  $(x_1,x_2,x_3)=(3,4,8)$.
\end{ex}

\begin{sagesilent}
  A = matrix([[0,1,0,0],[0,0,0,1],[0,0,0,0]])
  b = vector([0,0,1])
  M = A.augment(b, subdivide=True)
\end{sagesilent}

\begin{samepage}
  \begin{ex}
    The rref system
    \[
    \sage{M}
    \]
    is inconsistent. The last equation is $0=1$, which is impossible! This
    system has \emph{no solutions}.
  \end{ex}
\end{samepage}

\begin{samepage}
  \begin{thm}
    A consistent rref system has
    \begin{description}
    \item[\emph{a unique solution}] if it has no free variables
    \item[\emph{infinitely many solutions}] if it has at least one free variable
    \end{description}
    An inconsistent rref system has no solutions.
  \end{thm}
\end{samepage}


\pagebreak
\section{Elementary Row Operations}

\begin{definition}
  There are three types of \emph{elementary row operations}:
  \begin{description}
  \item[\emph{row switching}:] a row can be switched with another row
    $R_i\leftrightarrow R_j$
  \item[\emph{row multiplication}:] a row can be multiplied by any nonzero
    scalar $c\cdot R_i\to R_i$
  \item[\emph{row addition}:] a row can be replaced by the sum of that row and a
    multiple of another row $R_i+c\cdot R_j\to R_i$
  \end{description}
\end{definition}

\begin{sagesilent}
  A = random_matrix(ZZ, 4, 4)
  b = vector([1,-1,2,3])
  M = A.augment(b, subdivide=True)
  N = copy(M)
  M.swap_rows(1,3)
\end{sagesilent}

\begin{ex}[row switching]
  \[
  \sage{N}\xrightarrow{R_2\leftrightarrow R_4}\sage{M}
  \]
\end{ex}

\begin{sagesilent}
  A = random_matrix(ZZ, 4, 5)
  b = vector([1,0,1,3])
  M = A.augment(b, subdivide=True)
  N = copy(M)
  M.rescale_row(2,3)
\end{sagesilent}

\begin{ex}[row multiplication]
  \[
  \sage{N}\xrightarrow{3\cdot R_3\to R_3}\sage{M}
  \]
\end{ex}

\begin{sagesilent}
  A = random_matrix(ZZ, 4, 5)
  b = vector([1,0,1,3])
  M = A.augment(b, subdivide=True)
  N = copy(M)
  M.add_multiple_of_row(2,1,-3)
\end{sagesilent}

\begin{ex}[row addition]
  \[
  \sage{N}\xrightarrow{R_3-3\cdot R_2\to R_3}\sage{M}
  \]
\end{ex}

\begin{samepage}
  \begin{thm}
    Performing an elementary row operation on an augmented matrix yields an
    equivalent system.
  \end{thm}
\end{samepage}


\begin{samepage}
  \begin{idea}
    To solve a given system, use elementary row operations to obtain a rref
    system.
  \end{idea}
\end{samepage}


\begin{samepage}
  \begin{thm}[Gau{\ss}-Jordan Elimination]
    Every augmented matrix is equivalent to a unique rref augmented
    matrix. Furthermore, its rref may be obtained via elementary row operations.
  \end{thm}
\end{samepage}

\begin{sagesilent}
  import gaussjordan
  A = matrix([[1,2],[3,4]])
  b = vector([3,0])
  M = A.augment(b, subdivide=True)
\end{sagesilent}

\pagebreak
\begin{ex}
  Find all solutions to the system
  \[
  \begin{array}{rcrcr}
    x    & + & 2\,y & = & 3 \\
    3\,x & + & 4\,y & = & 0
  \end{array}
  \]
\end{ex}
\begin{sol}
  Row reduce
  {\allowdisplaybreaks  
    \sage{gaussjordan.latex_reduction(A, M)}}%
  This shows that
  \[
  \rref\sage{M}=\sage{M.rref()}
  \]
  The rref of the system is consistent and has no free variables. Hence the
  unique solution is $(x,y)=(-6,\nicefrac{9}{2})$.
\end{sol}

\begin{sagesilent}
  var('x1 x2 x3')
  A = matrix([[0,4,1],[2,6,-2],[4,8,-5]])
  b = vector([2,3,4])
  M = A.augment(b, subdivide=True)
  u = matrix.column([x1, x2, x3])
  v = matrix.column([7/4*x3, 1/2-1/4*x3, x3])
  w1 = matrix.column([0, 1/2, 0])
  w2 = matrix.column([7/4, -1/4, 1])
\end{sagesilent}

\pagebreak
\begin{ex}
  Find all solutions to
  \[
  \begin{array}{rcrcrcr}
    &   & 4\,x_2 & + & x_3    & = & 2 \\
    2\,x_1 & + & 6\,x_2 & - & 2\,x_3 & = & 3 \\
    4\,x_1 & + & 8\,x_2 & - & 5\,x_3 & = & 4
  \end{array}
  \]
\end{ex}
\begin{sol}
  Row reduce
  {\allowdisplaybreaks
    \sage{gaussjordan.latex_reduction(A, M)}}%
  This shows that
  \[
  \rref\sage{M}=\sage{M.rref()}
  \]
  The rref is consistent with free variable $\Set{x_3}$ and dependent variables
  $\Set{x_1,x_2}$. The system has infinitely many solutions given by
  \[
  \sage{u}=\sage{v}=\sage{w1}+\sage{w2}x_3
  \]
\end{sol}

\pagebreak
\section{Homogeneous Systems}

\begin{definition}
  The system \eqref{eq:motivation} is \emph{homogeneous} if
  $b_1=b_2=\dotsb=b_m=0$. In a homogeneous system, the solution
  $x_1=x_2=\dotsb=x_n=0$ is the \emph{trivial solution}. All other solutions are
  called \emph{nontrivial solutions}.
\end{definition}

\begin{samepage}
  \begin{thm}[Theorem 1.1 in \cite{peterson}]
    A homogeneous system of $m$ equations and $n$ variables has infinitely many
    solutions if $m<n$.
  \end{thm}
\end{samepage}


\section{Rank and Nullity}

\begin{samepage}
  \begin{definition}
    The \emph{rank} of a consistent system is the number of pivots in its
    rref. Equivalently, the rank is the number of dependent variables.
  \end{definition}
\end{samepage}


\begin{samepage}
  \begin{thm}
    $\mathtt{rank}\leq\#\mathtt{variables}$ and
    $\mathtt{rank}\leq\#\mathtt{equations}$
  \end{thm}
\end{samepage}


\begin{samepage}
  \begin{thm}
    A consistent system has
    \begin{description}
    \item[\emph{a unique solution}] if $\mathtt{rank}=\#\mathtt{variables}$
    \item[\emph{infinitely many solutions}] if
      $\mathtt{rank}<\#\mathtt{variables}$
    \end{description}
  \end{thm}
\end{samepage}


\begin{samepage}
  \begin{definition}
    The \emph{nullity} of a consistent system is the number of nonpivot columns
    in its rref. Equivalently, the rank is the number of free variables.
  \end{definition}
\end{samepage}


\begin{samepage}
  \begin{thm}
    $\mathtt{rank}+\mathtt{nullity}=\#\mathtt{variables}$
  \end{thm}
\end{samepage}


\begin{samepage}
  \begin{thm}
    A consistent system has
    \begin{description}
    \item[\emph{a unique solution}] if $\mathtt{nullity}=0$
    \item[\emph{infinitely many solutions}] if $\mathtt{nullity}>0$
    \end{description}
  \end{thm}
\end{samepage}


\begin{sagesilent}
  import gaussjordan
  A = matrix([[2,0,1,-1],[0,1,2,1],[2,-1,-1,-2]])
  var('b1 b2 b3')
  b = vector([b1, b2, b3])
  M = A.change_ring(SR).augment(b, subdivide=True)
\end{sagesilent}

\begin{ex}
  How many solutions will a system with coefficient matrix
  \[
  A = \sage{A}
  \]
  have?
\end{ex}
\begin{sol}
  Row reduce
  {\allowdisplaybreaks  
    \sage{gaussjordan.latex_reduction(A, M)}}%
  The system is consistent if and only if $-b_1+b_2+b_3\neq 0$.

  Suppose $-b_1+b_2+b_3\neq 0\neq 0$, so that the system is consistent.

  Note that
  \[
  \mathtt{rank}=2<\#\mathtt{variables}=4
  \]
  so the system has infinitely many solutions.

  Alternatively,
  \[
  \mathtt{nullity}=2>0
  \]
  so the system has infinitely many solutions.

  Thus, any system with coefficient matrix $A$ will be inconsistent if
  $-b_1+b_2+b_3\neq 0$ and consistent with infinitely many solutions if
  $-b_1+b_2+b_3=0$.
\end{sol}

\bibliography{mybib.bib}{}
\bibliographystyle{plain}

\end{document}
